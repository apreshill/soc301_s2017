\documentclass[11pt]{exam}
\usepackage[]{graphicx}\usepackage[]{color}
%% maxwidth is the original width if it is less than linewidth
%% otherwise use linewidth (to make sure the graphics do not exceed the margin)
\makeatletter
\def\maxwidth{ %
  \ifdim\Gin@nat@width>\linewidth
    \linewidth
  \else
    \Gin@nat@width
  \fi
}
\makeatother

\usepackage{alltt}
%\usepackage{fullpage}
\usepackage{amsmath,amssymb,array}
\usepackage{color,graphics,epsfig, subfigure,comment,hhline}
\usepackage{mathrsfs}
\usepackage{enumerate}
\usepackage{amsthm}
\usepackage{cancel}
%\usepackage{mathtools}
%\usepackage{algorithmic}
%\usepackage{algorithm}
\usepackage{graphicx}
\usepackage{fancyvrb}
\usepackage[width=7in, paperheight=11.2in]{geometry}
\usepackage{bigints}
\usepackage{multicol}
%\usepackage[framed,numbered, autolinebreaks,useliterate]{mcode}
\usepackage{longtable,booktabs}

\usepackage[formats]{listings}
\lstdefineformat{R}{~=\( \sim \)}
\lstset{basicstyle=\ttfamily,format=R}

\newcommand{\mytilde}{$\sim$}

\newcommand{\prob}[1]{\mathbb{P}(#1)}
\newcommand{\E}[1]{\mathbb{E}(#1)}
\newcommand{\V}[1]{\mathbb{V}(#1)}
\IfFileExists{upquote.sty}{\usepackage{upquote}}{}

\definecolor{fgcolor}{rgb}{0.345, 0.345, 0.345}
\newcommand{\hlnum}[1]{\textcolor[rgb]{0.686,0.059,0.569}{#1}}%
\newcommand{\hlstr}[1]{\textcolor[rgb]{0.192,0.494,0.8}{#1}}%
\newcommand{\hlcom}[1]{\textcolor[rgb]{0.678,0.584,0.686}{\textit{#1}}}%
\newcommand{\hlopt}[1]{\textcolor[rgb]{0,0,0}{#1}}%
\newcommand{\hlstd}[1]{\textcolor[rgb]{0.345,0.345,0.345}{#1}}%
\newcommand{\hlkwa}[1]{\textcolor[rgb]{0.161,0.373,0.58}{\textbf{#1}}}%
\newcommand{\hlkwb}[1]{\textcolor[rgb]{0.69,0.353,0.396}{#1}}%
\newcommand{\hlkwc}[1]{\textcolor[rgb]{0.333,0.667,0.333}{#1}}%
\newcommand{\hlkwd}[1]{\textcolor[rgb]{0.737,0.353,0.396}{\textbf{#1}}}%
\let\hlipl\hlkwb

\usepackage{framed}
\makeatletter
\newenvironment{kframe}{%
 \def\at@end@of@kframe{}%
 \ifinner\ifhmode%
  \def\at@end@of@kframe{\end{minipage}}%
  \begin{minipage}{\columnwidth}%
 \fi\fi%
 \def\FrameCommand##1{\hskip\@totalleftmargin \hskip-\fboxsep
 \colorbox{shadecolor}{##1}\hskip-\fboxsep
     % There is no \\@totalrightmargin, so:
     \hskip-\linewidth \hskip-\@totalleftmargin \hskip\columnwidth}%
 \MakeFramed {\advance\hsize-\width
   \@totalleftmargin\z@ \linewidth\hsize
   \@setminipage}}%
 {\par\unskip\endMakeFramed%
 \at@end@of@kframe}
\makeatother

\definecolor{shadecolor}{rgb}{.97, .97, .97}
\definecolor{messagecolor}{rgb}{0, 0, 0}
\definecolor{warningcolor}{rgb}{1, 0, 1}
\definecolor{errorcolor}{rgb}{1, 0, 0}
\newenvironment{knitrout}{}{}

\begin{document}

%\printanswers
%\unframedsolutions
\noprintanswers   %%NOT TO PRINT ANSWERS

\header{\Large SOC 301 - Social Statistics \\ \large Sampling Matching \\ }{}{\Large Name: \underline{\hspace*{3in}}\\ 
\large April 4, 2017 \\}

\headrule
\footer{}{}{}

\renewcommand{\theenumi}{\Alph{enumi}}

Match the term with its correct definition.

\begin{enumerate}
	\item bias
	\item sampling
	\item sample
	\item generalizability
	\item parameter
	\item statistic
	\item population
	\item representative sample
\end{enumerate}

\begin{questions}

\question The  \fillin is the (usually) large pool of observations (instances of observational units) that we are interested in.

\question A  \fillin  is a smaller collection of observations (instances of observational units) that is selected from the larger pool.

\question  \fillin  refers to the process of selecting observations from a population.  There are both random and non-random ways this can be done.

\question A sample is said be a  \fillin  if the characteristics of observational units selected are a good approximation of the characteristics from the original population.

\question \fillin corresponds to a favoring of one group in a population over another group.

\question \fillin refers to the largest group in which it makes sense to make inferences about from the sample collected.  This is directly related to how the sample was selected.

\question A 
\fillin is a calculation based on one or more variables measured in the population.  Parameters are almost always denoted symbolically using Greek letters such as $\mu$, $\pi$, $\sigma$, $\rho$, and $\beta$.

\question A \fillin is a calculated based on one or more variables measured in the sample.  Parameters are usually denoted by lower case Arabic letters with other symbols added sometimes.  These include $\bar{x}$, $\hat{p}$, $s$, $r$, and $b$. 

\end{questions}

\end{document}


